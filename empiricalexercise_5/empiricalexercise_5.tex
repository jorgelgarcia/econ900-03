%Input preamble
\input{preamble}
\let\counterwithout\relax
\let\counterwithin\relax
\definecolor{maroon}{HTML}{4B0082}

\begin{document}

\noindent \textbf{900-03: Econometrics II: Advanced Regression Analysis and Empirical Design}\\
\noindent \textbf{Wednesdays and Fridays, 10:00 to 11:30}\\
\noindent \textbf{Location: TBA}\\
\noindent Instructor: Jorge Luis García \\
e-mail: jlgarci@clemson.edu\\

\noindent \textbf{Empirical Exercise 5.}\\

\noindent \textbf{IV in Practice.} I posted data on a school-admission system. Individuals are either admitted or not to school based on a categorical lottery that takes place yearly. The category of their lottery is assigned according to their grades in the previous grade. There are as many lotteries as categories. The dataset explains the rest of the variables. The objective of this exercise is to investigate the causal effect of admission to school on labor income (in logs).

\begin{enumerate}
\item Suppose that your plan is to use the variable $z$ as an instrument for $d$ in a plain-vanilla IV setting. 
\begin{enumerate}
\item Discuss if the assumptions of the instrumental-variable setting hold. 
\item Is the instrument relevant? 
\item Estimate the causal effect of interest using IV. 
\item What is the number of compliers by gender? Interpret. 
\item Is the instrumental-variables estimate equal to the ATT? 
\item Are the distributions of $y_{i0}$ and $y_{i1}$ identified? If so, estimate them. 
\item Interpret the distributions of $y_{i0}$ and $y_{i1}$ for always and never takers. 
\end{enumerate}
\item Now, note that the lottery is only exogenous within category and year.
	\begin{enumerate}
		\item Estimate category-year-wise LATE parameters and combine them to one LATE. 
		\item Use the results in $\textbf{Problem 2}$ to obtain one LATE and comment on the difference with the LATE that you estimated in (a). 
	\end{enumerate}
\end{enumerate} 

\noindent \textbf{Regression Discontinuity Design.} I have posted the data related in \citet{angrist_using_1999}. The purpose of this exercise is for you to design an empirical strategy to asses the effect of class size on test scores, using RDD. I will give you some pointers: 

\begin{enumerate}
\item Use OLS without and with controls (use the percentage of disadvantage children and enrollment as controls throughout) to understand the effect of class size on math. 
\item Limit the sample to schools with enrollment between 20 and 60 students.
\item Predict large class size based on the first discontinuity (40 students). 
\item Estimate the effect of interest using OLS and assuming that there is a sharp discontinuity, without and with controls. 
\item Repeat this exercise using a local-linear regression and use bootstrap to estimate standard errors. 
\item Investigate a fuzzy RDD design using the command rdrobust. 
\item Use instrumental variables. 
\item Provide no-manipulation evidence. 
\item Provide a placebo check. 
\end{enumerate} 

\bibliographystyle{chicago}
\bibliography{bib}
\end{document}
