%Input preamble
\input{preamble}
\let\counterwithout\relax
\let\counterwithin\relax
\definecolor{maroon}{HTML}{4B0082}

\begin{document}

\noindent \textbf{900-03: Econometrics II: Advanced Regression Analysis and Empirical Design}\\
\noindent Instructor: Jorge Luis García \\
e-mail: jlgarci@clemson.edu\\

\noindent \textbf{Problem Set 3.}\\

\noindent \textbf{Submission instructions for empirical part:} Push the code for each empirical problem separately with the name problem``numberofproblem"\_``yourlastname".do. Do so to a folder in the repository called problemset3 (you need to create that folder in your local). \textit{Do not e-mail me your code}.  \textbf{Submission instructions for theory part:} Place a .pdf with your answer in the Box folder econ900/econ900-03/problemset3. \textit{Do not e-mail your pdf}.\\

\noindent \textbf{Due date:} 11/14/2020 at 8:00 p.m.\\

\noindent \textbf{Problem 1. Matching, Theory.} Let $y_{i}^d$ be the outcome of individual $i$ when fixed to treatment ($D_{i} = 1$) or control $(D_{i} = 0)$. Let $y_{i}$ be the observed outcome and $D_{i}$ a treatment indicator. 

\begin{enumerate}
\item What does the notation mean? Specifically, what is the difference between the upper and lower cases of ``d.''
\item What is the difference between fixing and conditioning? 
\item When, if ever, would you not care about this difference? 
\item What are the standard matching assumptions? Use $\bm{x}_{i}$ to denote the vector of observed characteristics.
\item Explain these assumptions. 
\item Given these assumptions, propose an estimator for $\mathbb{E} \left[ y_{i}^1 - y_{i}^0 \right]$. 
\item Propose an inference procedure. 
\item Show that the matching assumptions imply that $D_{i}$ is independent of the potential outcomes conditional on $\Pr \left[ D_{i} = 1 | \bm{x}_{i} \right]$ (propensity score, henceforth). 
\item Interpret the result of your proof.  
\item Propose an estimator of $\mathbb{E} \left[ y_{i}^1 - y_{i}^0 \right]$ that uses the propensity score.
\item Propose an inference procedure. 
\item A speaker in the seminar has a linear regression and calls their empirical strategy ``matching.'' Professor Simon calls them out and tells them that they are ``simply running OLS.'' Map the notation in this problem into a linear regression  to defend the speaker from Professor Simon. Is the speaker ``overselling'' their empirical strategy or is Professor Simon in a mood and that is why he is calling the speaker out? 
\end{enumerate}

\noindent \textbf{Problem 2. Matching, Practice.} Your outcome of interest is test scores and your policy binary variable of interest is homework completion. You have data on test scores in the year that you are analyzing, homework completion, number of class absences, gender, age, test scores in the previous year.

\begin{enumerate} 
\item What is the parameter of interest for abolishing homework? 
\item What is the parameter of interest for making homework mandatory?  
\item What is the parameter of interest for comparing the effect of an extra hour of schooling relative to doing homework?  
\item Suppose that you can only use matching. What fundamental assumption would allow you to study how would the students who currently do not do homework do if they did the homework? 
\item What is the parameter of interest?
\item What are the counterfactual outcomes of interest?
\item Provide an OLS estimator of the parameter of interest.
\item Provide examples of violations to the fundamental assumption biasing the estimate upwards.
\item Provide examples of violations to the fundamental assumption biasing the estimate downwards.
\end{enumerate}

\noindent \textbf{Problem 3. TFU is not only for Price Theory.}

\begin{enumerate} 
\item The Conditional Independence Assumption is untestable. 
\item When solving missing data problems, hot-decking and cold-decking are based on the same assumptions as IPW.
\item If you have a randomized control trial and you want to obtain the ATE, you can either (i) use an estimator based on means; or (ii) regress the outcomes of interest on a treatment indicator. It is never correct to include an intercept in the regression of option (ii). 
\item If you have a randomized control trial and you want to obtain the ATE, you can either (i) use an estimator based on means; or (ii) regress the outcomes of interest on a treatment indicator. It is never incorrect to include controls in the regression of option (ii). 
\item The estimates obtained from IPW weighted regressions are numerically equivalent to estimates obtained from unweighted OLS regressions. 
\item Data from experiments are always better for answering policy questions if compared to observational data. 
\end{enumerate} 

\bigskip

\noindent \textbf{Problem 4. IPW Equivalence.} Consider a data-generating process of the form $\left( Y_{i}^0, Y_{i}^0, D_i, \bm{x}_i \right)$, where the usual notation applies. Suppose that $\bm{x}_i$ contains all discrete categories, with typical element $j$ and index set $\mathcal{J}$. You want to implement an IPW scheme to obtain estimates of the ATE, the ATT, and the ATU. Show that: 

\begin{enumerate} 
\item The two following estimators are equivalent: (i) the weighted sum of the $j$-wise treatment-control average difference; and (ii) a weighted regression. \textit{[Hint: State the weights for each parameter and compute the estimators.]}
\end{enumerate}

\noindent \textbf{Problem 5. More IPW.} Suppose that the data-generating process takes the form: 

\begin{eqnarray} 
y_{i}^0 &=& 2 + 2 \cdot \bm{x}_{i1} \cdot \bm{x}_{i2} + \varepsilon_{i}^0 \nonumber \\ 
y_{i}^1 &=& 3 + 2 \cdot \bm{x}_{i1} \cdot \bm{x}_{i2} + \varepsilon_{i}^1, 
\end{eqnarray} 

\noindent with $\mathbb{E}\left[  \varepsilon_{i}^0 \right] = \mathbb{E}\left[  \varepsilon_{i}^1 \right] = 0$ and where the standard notation applies. 

\begin{enumerate} 
\item State the conditional independence assumption and write its implications with respect to the distributions of the counterfactual outcomes and unobserved components. Assume that the CIA holds henceforth. 
\item Calculate the ATE? 
\item Is the ATE that you wrote down an estimate or a true value? 
\item The joint distribution of treatment take up and observed characteristics is

\begin{center}
\begin{tabular}{ccccccc} \toprule
& \multicolumn{3}{c}{\underline{$D_{i} = 1$}} &  \multicolumn{3}{c}{\underline{$D_{i} = 0$}} \\
& $\bm{x}_{i1} = 0$ & $\bm{x}_{i1} = 1$ & $\bm{x}_{i1} = 2$ & $\bm{x}_{i1} = 0$ & $\bm{x}_{i1} = 1$ & $\bm{x}_{i1} = 2$ \\
$\bm{x}_{i2} = 0$ & .1 & .05 & .1 & .1 & .1& .05 \\ 
$\bm{x}_{i2} = 1$ & .1 & .05 & .1 & .1 & .05 & 1 \\  \bottomrule
\end{tabular}
\end{center}

\begin{enumerate} 
\item A researcher is interested in estimating the ATE and writes down the following linear model 

\begin{equation}
y_i = \beta_0 + \beta_1 \bm{x}_{i1} + \beta_2 \bm{x}_{i2} + \beta_3 D_{i} + \eta_{i}
\end{equation}

\item Show (empirically) that the OLS estimator of $\beta_3$ that he is setting up is an inconsistent estimate of the ATE. 
\item Suppose that the researcher estimates the propensity score using the following linear model: 

\begin{equation}
D_i = \alpha_0 + \alpha_1 \bm{x}_{i1} + \alpha_2 \bm{x}_{i2} + \upsilon_{i} 
\end{equation}

\noindent to then implement an IPW scheme. Show (empirically) that this model produces an inconsistent estimate of the ATE. 
\end{enumerate}
\item Interpret the exercise and answer the following: Would fully saturating the model have helped the researcher? \textit{[Hint: Fully saturating means including mutually exclusive and exhaustive dummy variables describing observed characteristics.]}
\end{enumerate} 

\noindent \textbf{Problem 6. IPW Equivalence.} Go back to the IPW application discussed in class. Write a code that does the following: 
\begin{enumerate}
	\item Calculate the ATT using the weighted-sum method. 
	\item Calculate the ATT using the regression method.
	\item Calculate the ATU using the weighted-sum method. 
	\item Calculate the ATU using the regression method.
	\item Plot the distribution of the j-wise ATT and j-wise ATU in the same plot.
\end{enumerate}


\end{document}