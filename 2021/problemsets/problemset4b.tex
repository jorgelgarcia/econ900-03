%Input preamble
\input{preamble}
\let\counterwithout\relax
\let\counterwithin\relax
\definecolor{maroon}{HTML}{4B0082}

\begin{document}

\noindent \textbf{900-03: Econometrics II: Advanced Regression Analysis and Empirical Design}\\
\noindent Instructor: Jorge Luis García \\
e-mail: jlgarci@clemson.edu\\

\noindent \textbf{Problem Set 4.}\\

\noindent \textbf{Submission instructions for theory part (Problems 1 to 6):} Place a pdf with your answer in the Box folder econ900/econ900-03/problemset4. \textit{Do not me e-mail your pdf}.\\

\noindent \textbf{Submission instructions for empirical part (Problem 7):} Push the code for your empirical problem separately with the name problem``numberofproblem"\_``yourlastname".do. Do so to a folder in the repository called problemset4 (you need to create that folder in your local). \textit{Do not e-mail me your code}.\\

\noindent \textbf{Due date:} 11/22/2020 at 2:00 p.m.\\

\noindent \textbf{Problem 1. The Differences in Difference-in-Differences.}\\
\begin{enumerate}
\item Write down the linear model allowing you to estimate the ATT using difference-in-differences in the baseline two-group, two-period framework. 
\item Suppose that the error term in your model is an individual-level lottery. Express each of the parameters of your model in terms of expectations and answer: What difference does each parameter identify?\\
\end{enumerate}

\noindent \textbf{Problem 2. Difference-in-Difference-in-Differences.}\\
\begin{enumerate} 
\item Write down the linear model allowing you to estimate difference-in-difference-in-differences. Assume that there are two periods ($t$), two treatment statuses $d$, and two locations $l$. In one location, there is not treatment at all.
\item Suppose that the error term in your model is an individual-level lottery. Express each of the parameters of your model in terms of expectations and answer: What difference does each parameter identify and under what assumptions(s)?\\
\end{enumerate}

\noindent \textbf{Problem 3. Inference in Difference-in-Differences.}\\
\noindent Consider the model
\begin{equation} 
y_{ikt} = \beta D_{kt} + \lambda_k + \lambda_t + \varepsilon_{ikt}. \label{eq:diff2}
\end{equation}

\noindent Suppose that $\mathbb{E}\left[\varepsilon_{ikt}\varepsilon_{jkt}\right] \neq 0$ for $i \neq j$. A researcher proposes an inference procedure based on $\varepsilon_{ikt}:= \nu_{kt} + \tilde{\varepsilon}_{ikt}$ with $\tilde{\varepsilon}_{ikt}$ being assigned by an individual-level lottery. 

\begin{enumerate}
\item Is the researcher's procedure based on a parametric assumption? 
\item Propose an estimator for the variance of $\beta$ for the researcher. 
\item Give them a concrete alternative using a Huber-White variance estimator. 
\item Explain to them if this relaxes their assumptions, if they are making any. 
\item Explain to them how to obtain a variance estimate using a block-$k$ bootstrap procedure. 
\item Explain to them if this relaxes their assumptions, if they are making any.  
\item Would they be able to improve their inference if grouping their model at the $k$ level and then obtaining different variance estimators?
\end{enumerate} 

\noindent \textbf{Problem 4. The Average and Local Average Treatment Effects.} Let a linear model be 
\begin{eqnarray}
y_{i} &=& y_{i0} + D_{i} \cdot \left( y_{i1} - y_{i0} \right) \nonumber \\ 
        &=& \beta_0 + \beta_1 D_{i} + \varepsilon_{i}.  
\end{eqnarray}
\begin{enumerate} 
\item Use the material that we discussed in class to interpret each of the elements in this model. 
\item State the assumptions of the instrumental-variable setting for this model. 
\item Show that the ATT and the LATE are not the same in general and explain why. 
\item When is the LATE equal to the ATE? Provide a formal argument. 
\end{enumerate}

\noindent \textbf{Problem 5. Residual Instrumental Variables.} Consider an extension of the model in \textbf{Problem 4}, where the researcher aims to control for a vector of observed characteristics $\bm{x}_{i}$. 

\begin{enumerate} 
\item Why would a researcher aim to control for a vector of observed characteristics if the assumptions of the instrumental-variable setting hold? 
\item Is there another scenario where the researcher should consider a vector of observed characteristics?
\item Adapt and state the instrumental-variable setting assumption to consider the vector $\bm{x}_{i}$. 
\item Derive an expression for the $\text{LATE} \left( x \right) $. 
\item Show and interpret the following 
\begin{eqnarray}
\text{LATE} &=& \mathbb{E} \left[ y_{i1} - y_{i0} | \text{compliers} \right] \nonumber \\ 
                   &=& \mathbb{E} \left[ \frac{ \text{LATE} \left( \bm{x}_i \right) \cdot  \Pr \left( \text{compliers} | \bm{x}_{i} \right) } {  \Pr \left( \text{compliers} \right) } \right]. 
\end{eqnarray}

\item Assume that $\bm{x}_{i}$ is composed of a set of (binary) indicators and that the instrument, $z_{i}$, is binary. Note that the excluded instruments are now $z_{i}$ and all of the interactions of $\bm{x}_{i}$ with $z_{i}$. Think of a fully-saturated model. Interpret the following (you do not need to show it)
\begin{eqnarray} 
\beta^{\text{TSLS}} &=& \mathbb{E} \left[ \text{LATE} \left( \bm{x}_{i} \right)   \frac{\var \left(  \Pr \left(  D_{i} = 1 | z_{i}, \bm{x}_{i} | \bm{x}_{i} \right)   \right)}{ \mathbb{E} \left[ \var \left(  \Pr \left(  D_{i} = 1 | z_{i}, \bm{x}_{i} | \bm{x}_{i} \right)   \right) \right] }  \right]. 
\end{eqnarray}
\end{enumerate}

\noindent \textbf{Problem 6. Selection and Instrumental Variables.} Let the potential outcomes of a program be generated as follows
\begin{eqnarray} 
y_{i}^0 &=& \mu^0 + u_{i}^0 \\ \nonumber 
y_{i}^1 &=& \mu^1 + u_{i}^1, \nonumber 
\end{eqnarray}

\noindent where the usual notation applies ($ \mu^0$ and $ \mu^1$ are constants). If choosing treatment, individuals pay $c_{i}= k$. The selection to the program is based on gains: $D_{i} = 1$ iff $y_{i}^1 - y_{i}^0 - c_{i} \geq 0$.  
\begin{enumerate} 
\item Consider the linear model $y_{i} = \alpha + \beta_i \cdot D_{i} + \varepsilon_i$. What assumptions about the data-generating process does this model make?
\item Show that, in general, $\cov \left( \beta_i, \varepsilon_i \right) \neq 0$.
\item Define the ATE in this context.
\item Show that the OLS estimator of the ATE is not consistent. 
\item Describe a binary lottery allowing you to identify and estimate the average of $\beta_i$ using instrumental variables. 
\end{enumerate}

\noindent \textbf{Problem 7. Heckit Details.} Consider the empirical problem of incidental truncation covered in class (Nigerian data). 

\begin{enumerate}
\item Show (empirically) that the ML and Two-Step Heckman estimators provide approximately the same result. Why is this equivalence approximate?
\item Construct the inverse Mill's ratio and regress the wage on the relevant controls and the inverse Mill's ratio. Is this procedure equivalent to either of the procedures above?
\end{enumerate} 



\end{document}
