%Input preamble
\input{preamble}
\let\counterwithout\relax
\let\counterwithin\relax
\definecolor{maroon}{HTML}{4B0082}

\begin{document}

\noindent \textbf{900-03: Econometrics II: Advanced Regression Analysis and Empirical Design}\\
\noindent \textbf{Wednesdays and Fridays, 10:00 to 11:30}\\
\noindent \textbf{Location: TBA}\\
\noindent Instructor: Jorge Luis García \\
e-mail: jlgarci@clemson.edu\\

\noindent \textbf{Problem Set 4.}\\

\noindent \textbf{Problem 1. Selection and Instrumental Variables.} Let the potential outcomes of a program be generated as follows
\begin{eqnarray} 
y_{i0} &=& \mu_0 + u_{i0} \\ \nonumber 
y_{i1} &=& \mu_1 + u_{i1}, \nonumber 
\end{eqnarray}

\noindent where the usual notation applies. If choosing treatment, $d = 1$, individuals pay $c_{i}= k$. The selection to the program is based on gains: $D_{i} = 1$ iff $y_{i1} - y_{i0} - c_{i} \geq 0$.  
\begin{enumerate} 
\item Consider the linear model $y_{i} = \alpha + \beta_i \cdot D_{i} + \varepsilon_i$. What assumptions about the data-generating process does this model make? 
\item Show that, in general, $\cov \left( \beta_i, \varepsilon_i \right) \neq 0$. 
\item Define the ATE in this context. 
\item Show that the OLS estimator of the ATE is not consistent. 
\item Design a binary lottery allowing you to identify and estimate the average of $\beta_i$ using instrumental variables.
\end{enumerate}

\noindent \textbf{Problem 2. Normal, Polynomial Selection.} Allow for the following specificities to hold in the framework of \textbf{Problem 1}: 

\begin{eqnarray} 
\mu_1 \left( x \right) &=& a + b \cdot x + c \cdot x^2 \nonumber \\
\mu_0 \left( x \right) &=& m,
\end{eqnarray}

\noindent where $a,b,c$ and $m$ are real numbers and $x$ is a realization of the random variable $\bm{x_{i}}$. Economists usually call that random variable a regressor.

\begin{enumerate}
\item Let $u_{id} \sim \mathcal{N} \left( 0 , \sigma_d^ 2 \right)$. Derive the probability of selection into treatment. Do you need to make any additional assumptions? State them clearly.
\item Provide a strategy to identify and estimate the parameters of the model. 
\item Suppose that $c_i = k$ and that the analyst knows $k$. What parameters are identified?
\item How many observations do you need to estimate the decision model? 
\item With the minimum amount of observations, what are the standard errors of the estimates?
\item Suppose that $c_i$ varies across individuals and that the analyst knows it for many observations. What parameters are identified? 
\end{enumerate}

\noindent \textbf{Problem 3. Non-Parametric and Normal Generalized Roy Models.} Consider the following specification of the model in \textbf{Problem 1}:
\begin{eqnarray} 
y_{i0} &=& \mu_0 \left( x \right)  + u_{i0}  \nonumber \\
y_{i1} &=& \mu_1 \left( x \right)  + u_{i1}   \nonumber \\
D_{i}^* &=& \mu_{z} \left( z \right) - v_i     \nonumber \\
D_{i} &=& \bm{1} \left[ D_{i}^* \geq 0 \right], 
\end{eqnarray}

\noindent where the usual notation applies. 
\begin{enumerate} 
\item Describe and compare the program evaluation and selection problems. 
\item Consider a general, non-parametric selection model and derive $\mathbb{E} \left[ y_{i} | D_i = D, X = x, Z= z \right]$ for $d = 0,1$. 
\item Derive the ATE, ATT, and ATU 
\item Provide identification conditions. 
\item Consider a normal selection model where $u_{id} \sim \mathcal{N} \left( 0, \sigma_d^2 \right)$ for $d = 0,1$, $v_i \sim \mathcal{N} \left( 0, \sigma_v^2 \right)$, $\cov \left( u_{i0}, u_{i1} \right) = \sigma_{01}$,  and $\cov \left( u_{id}, v_{i} \right) = \sigma_{dv}$ for $d = 0,1$. Derive $\mathbb{E} \left[ y_{i} | D_i = D, X = x, Z= z \right]$ for $d = 0,1$. 
\item Derive the ATE, ATT, and ATU
\item Provide identification conditions. 
\end{enumerate}








\end{document}
