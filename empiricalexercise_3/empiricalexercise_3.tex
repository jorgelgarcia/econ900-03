%Input preamble
\input{preamble}
\let\counterwithout\relax
\let\counterwithin\relax
\definecolor{maroon}{HTML}{4B0082}

\begin{document}

\noindent \textbf{900-03: Econometrics II: Advanced Regression Analysis and Empirical Design}\\
\noindent \textbf{Wednesdays and Fridays, 10:00 to 11:30}\\
\noindent \textbf{Location: TBA}\\
\noindent Instructor: Jorge Luis García \\
e-mail: jlgarci@clemson.edu\\

\noindent \textbf{Empirical Exercise 3.}\\

\noindent \textbf{Matching and Experimental Methods.} 
Download the data on the problem set's folder. The data includes three samples: the sample from the National Supported Work (NSW) demonstration, a sample from the Current Population Survey, and a sample of the Panel Study of Income Dynamics (PSID). The variables sample and treated identify the sample and random assignment to the demonstration program for the relevant sample.
\begin{enumerate} 
\item In the NSW sample, what parameter can you identify exploiting randomization?\\
\noindent \textbf{Answer:} The answers so far directly imply that randomization identifies the average treatment effect.\\

\item What other parameters could you identify with more assumptions and what are these other assumptions?\\
\noindent \textbf{Answer:} With zero additional assumptions, randomization identifies the average treatment on the treated (ATT) and on the untreated (ATU) because if $\left( y_{i}^0, y_{i}^1 \right) \independent D_{i}$, $\text{ATE} = \text{ATT} = \text{ATU}$.\\
\item Write down a Stata program that does the following: 
	\begin{enumerate} 
		\item Verify that the randomization protocol worked out well in the NSW demonstration.\\ 
		\noindent \textbf{Answer:} See problem5\_solution.do (note explanations within file).
		\item Estimate the parameter in 2. Use an estimator based on means.\\ 
		\noindent \textbf{Answer:} See problem5\_solution.do (note explanations within file).\\
		\item What assumption would suffice for you to use the CPS sample as a control group?\\ 
		\noindent \textbf{Answer:} Conditional independence would make CPS a valid control group because there is no treatment in the CPS sample. Recall the first and second lectures to relate this to structural invariance across samples.\\
		\item Estimate the parameter in 2 using CPS as the control group. Use an estimator based on means.\\ 
		\noindent \textbf{Answer:} See problem5\_solution.do (note explanations within file).\\
		\item Estimate this parameter again using OLS to improve your estimator with respect to the previous estimator.\\ 
		\noindent \textbf{Answer:} See problem5\_solution.do (note explanations within file).\\
		\item Think about the covariates that you would use to investigate balance between the treatment and the CPS sample. Test balance.\\ 
		\noindent \textbf{Answer:} See problem5\_solution.do (note explanations within file).\\
		\item Estimate the parameter in 2. using nearest neighbor propensity score matching, propensity score, and local-linear regression. [Hint: ssc install psmatch2]\\ 
		\noindent \textbf{Answer:} See problem5\_solution.do (note explanations within file).
	\end{enumerate}
\end{enumerate}

\end{document}