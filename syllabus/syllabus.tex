%Input preamble
\input{preamble}
\let\counterwithout\relax
\let\counterwithin\relax
\definecolor{maroon}{HTML}{4B0082}

\begin{document}

\noindent \textbf{900-03: Econometrics II: Advanced Regression Analysis and Empirical Design}\\
\noindent \textbf{Wednesdays and Fridays, 10:00 to 11:30}\\
\noindent \textbf{Location: TBD}\\
\noindent Instructor: Jorge Luis García \\
e-mail: jlgarci@clemson.edu\\

\onehalfspacing
\noindent \textbf{Class Format and Regulations.} The class is in person, unless the university indicates that an online-class protocol is to be implemented. Regardless of university policies, the class will be available online in \textit{live} format. A \textit{Zoom} link for each class is posted on Canvas.\\

\noindent \textbf{COVID-19 Policy:} If Covid-19 policies are reimplemented, an update on the class format will be provided immediately.\\

\noindent \textbf{Face Coverings:} When in-person classes take place, you are expected to wear an appropriate face covering (mask) and social distance as dictated by the university policies. If  you do not have a face covering, refuse to wear an approved face covering without valid accommodation, or violate social-distancing policies, I will ask you to leave the academic space and report your actions to the Office of Community and Ethical Standards as a violation of the Student Code of Conduct.\\

\noindent \textbf{Class Schedule.} The first class meets on 10/13/2021. The last regular-schedule class meets on 11/19/2021. There are three additional lectures on 11/08/2021, 11/15/2021, and 11/22/2021 from 08:00 a.m. to 09:30 a.m. This totals 15 classes of one hour and a half. The final exam is on 11/23/2021 from 8:30 a.m. to 8:30 p.m. (take-home exam).\\

\noindent \textbf{Office Hours with Instructor.} Fridays from 10:00 a.m. to 11:30 a.m., \textit{by Zoom.} The link for the weekly office hours is posted on Canvas. The office hours will be open so that anyone enrolled in the class can join them and stay connected during the time indicated. If you require a private time-slot, please e-mail me so that we can schedule it. The additional lectures will have an attached schedule of office hours, in the same schedule.\\

\noindent \textbf{Teaching Assistant Information.} Micah Thomas. E-mail: micah7@clemson.edu. Office Hours: TBD.\\

\noindent \textbf{Course Description and Learning Outcomes.} This course is a primer on advanced linear-regression and empirical design. With the basic regression expertise of Econ-900-02, you will be able to think about designing empirical strategies using modern econometric tools.\\

\noindent \textbf{Prerequisites.} Econ-900-02. \\

\noindent \textbf{Required Readings and Material.}  I do not strictly follow a single textbook. I recommend that you acquire a graduate-level econometrics textbook such as \citet{greene_econometric_2008}. I also recommend that you download the textbook \citet{HansenEconometrics2021}. Both are basic references for this class. I will post on Canvas notes for each lecture.\\

\noindent \textbf{Class Structure and Learning Environment.} The schedule of the class is the following: 
	\begin{itemize}
		\item 10:00 to 10:10: Floor open for questions on previous session (on the date of the first lecture, syllabus discussion). 
		\item 10:10: to 11:10: Discussion of class material. 
		\item 11:10 to 11:30: Empirical exercise (Stata) related to the weekly class content. This part will not take place every class. In classes without this part, I will discuss material for a longer period of time.
	\end{itemize}


\noindent \textbf{Empirical Exercises.} The data for the empirical exercises that I will perform in class will be posted on Box.\\

\noindent \textbf{Assessments and Grading Policies.} The class will be graded according to the following assessments of your learning (which are detailed below):
\begin{itemize}
\item Problem Sets: 20\% of the final grade.
\item Final Exam: 75\% of the final grade.
\item Class Participation: 5\% of the final grade.
\end{itemize}

\noindent There is no curve in this class. This is the mapping between scores and letter grades: 95-100: A; 87-94: A-; 80-86: B+; 72-79: B; 65-71: B-; 59-64: C+; 55-59: C; 45-55: C-; $<$ 45: F.\\ 

\noindent \textbf{Team Work.} All work in this class is individual. If I catch you cheating in either a problem set or a test, you will automatically get an F. This is not a joke.\\

\noindent \textbf{Problem Sets.} I will problem sets, approximately every other week, on Fridays. The problem sets are due one week after they are assigned at the beginning of class. You have one week to complete this problem set. The problem sets will be 80\% written and 20\% empirical (similar to the class exercises). The problem sets will be graded on completion and the solutions will be posted on Canvas a couple of hours after they are due.\\

\noindent \textbf{Final Exam.} There is a final written exam on 11/22/2021. The exam is from 08:30 a.m. to 08:30 p.m. (take-home exam).\\

\noindent \textbf{Class Participation.} I will discuss the material for each of the topics. I will ask you to participate. This item of the grading is meant to ensure that you come prepared to class and participate of the discussion. Answers and discussion will grant you these points. 0 participation points will be assigned to people who miss more than 20\% of the lectures.\\

\noindent \textbf{Late Assignments.} The due date for each of the problem sets is set to be on Fridays at the beginning of each class. Late assignments are not accepted under any circumstance.\\

\noindent \textbf{Course Feedback.} Your feedback on the class material and dynamics is always welcome at anytime. Giving feedback to a professor is not disrespectful if you find the adequate way to do it. Come to my office hours or send me an e-mail with your feedback at any time during the semester. \textbf{If 90\% of the group completes the Canvas final course evaluation, I will give everyone 5 points towards the final grade.}\\

\noindent \textbf{e-mail Response Policy.} You can expect responses to e-mail inquiries within 24 hours.\\

\noindent \textbf{Course Outline.}
\begin{enumerate}
\item Models for binary-response dependent variables
\item Models for panel data
\item Selection on observables 
\item Inverse probability weighting 
\item Selection on unobservables 
\item Difference-in-differences
\item Event studies
\item Instrumental variables 
\item Regression discontinuity 
\item General method of moments 
\end{enumerate}
\noindent \textbf{Important Matters.}
\begin{itemize}
\item \textbf{Instructor's Absence.} If class does not meet due to instructor's absence or to inclement weather or any other event, a virtual lecture will be posted on Canvas. If the instructor is not in the classroom at 10:10, students may leave the classroom and expect no lecture to be held.
\item \textbf{Student Accessibility Services.} Clemson University values the diversity of our student body as a strength and a critical component of our dynamic community. Students with disabilities or temporary injuries or conditions may require accommodations due to barriers in the structure of facilities, course design, technology used for curricular purposes, or other campus resources. Students who experience a barrier to full access to this class should let the professor know, and make an appointment to meet with a staff member in Student Accessibility Services as soon as possible. You can make an appointment by calling 864-656-6848, by e-mailing studentaccess@lists.clemson.edu, or by visiting Suite 239 in the Academic Success Center building. Appointments are strongly encouraged---drop-ins will be seen if at all possible, but there could be a significant wait due to scheduled appointments. Students who receive Academic Access Letters are strongly encouraged to request, obtain and present these to their professors as early in the semester as possible so that accommodations can be made in a timely manner. It is the student's responsibility to follow this process each semester.

\item \textbf{Academic Continuity Plan for this Class.}
Clemson has developed an academic continuity plan for academic operations. Should University administration officially determine that the physical classroom facility is not available, class will be conducted in a virtual (online) format. The University issues official disruption notifications through email/ www/ test notification / social media. When notified, use one of the following links to navigate for Clemson Canvas, where you will find important information about how we will conduct class:

\begin{itemize}
\item Primary access: www.clemson.edu/Canvas
\item Secondary access link, if needed: https://clemson.instructure.com/
\item You can also use the Canvas Student App.
\end{itemize}

\item \textbf{Copyright.} All materials found in this course are strictly for the use of students enrolled in this course and for purposes associated with this course; they may not be retained or further disseminated. Clemson students, faculty, and staff are expected to comply fully with institutional copyright policy as well as all other copyright laws.

\item \textbf{Privacy Policy.} This course is designed with your privacy in mind. If, however, you feel that an assignment or technology tool undermines your right to privacy, please contact me immediately. We will work together to determine an alternative assignment that will help you achieve the course learning outcomes.

\item \textbf{Academic Integrity.} As members of the Clemson University community, we have inherited Thomas Green Clemson's vision of this institution as a "high seminary of learning." Fundamental to this vision is a mutual commitment to truthfulness, honor, and responsibility, without which we cannot earn the trust and respect of others. Furthermore, we recognize that academic dishonesty detracts from the value of a Clemson degree. Therefore, we shall not tolerate lying, cheating, or stealing in any form. A simple definition of plagiarism is when someone presents another person's words, visuals, or ideas as his or her own. The instructor will deal with plagiarism on a case-by-case basis. I will use, at my discretion, the Plagiarism Resolution Form. All infractions of academic dishonesty will be reported to Undergraduate Studies for resolution through that office.

\item \textbf{Academic Grievances.} Students are advised to visit the Ombuds' Office. After discussion with the undergraduate academic ombudsman, students should contact Undergraduate Studies (656-3022) for assistance filing official paperwork.

\item \textbf{Non-Discrimination.} Clemson University is committed to providing a higher education environment that is free from sexual discrimination. Therefore, if you believe you or someone else that is part of the Clemson University community has been discriminated against based on sex, or if you have questions about Title IX, please contact the Title IX Coordinator, Alesia Smith, who also serves as the Executive Director of Equity Compliance, at 110 Holtzendorff Hall, 864-656-3181 (voice) or 864-656-0899 (TDD). The Title IX Coordinator is the person designated by Clemson University to oversee its Title IX compliance efforts. Please consult the University's Title IX policy for full details.
\end{itemize}

\bibliographystyle{chicago}
\bibliography{econed.bib}

\end{document}