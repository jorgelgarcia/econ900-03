%Input preamble
\input{preamble}
\let\counterwithout\relax
\let\counterwithin\relax
\definecolor{maroon}{HTML}{4B0082}

\begin{document}

\noindent \textbf{900-03: Econometrics II: Advanced Regression Analysis and Empirical Design}\\
\noindent Instructor: Jorge Luis García. e-mail: jlgarci@clemson.edu\\
\noindent \textbf{Problem Set 2.} \\

\noindent \textbf{Submission instructions for empirical part:} Push the code for each empirical problem separately with the name problem``numberofproblem"\_``yourlastname".do. Do so to a folder in the repository called problemset2 (you need to create that folder in your local). \textit{Do not e-mail me your code}.  \textbf{Submission instructions for theory part:} Place a .pdf with your answer in the Box folder econ900/econ900-03/problemset2. \textit{Do not e-mail your pdf}.\\

\noindent \textbf{Due date:} 11/6/2020 at 8:00 p.m.\\

\noindent \textbf{Grading:} Each item of each question of each problem is worth one point. This problem set is not graded on completion. If you copy-down solutions found online, your grade for the problem sets of all of the course will be 0.\\

\noindent \textbf{Problem 1. Problem 1 of Chapter 9 of \citet{greene_econometric_2008}.}\\

\noindent \textbf{Problem 2. Problem 2 of Chapter 9 of \citet{greene_econometric_2008}.}\\

\noindent \textbf{Problem 3. Problem 5 of Chapter 9 of \citet{greene_econometric_2008}.}\\


\noindent \textbf{Problem 4. Application 1 of Chapter 9 of \citet{greene_econometric_2008}. In d., use an F-test instead of a Lagrange-multiplier test.}\\

\noindent I am referring to the numbering in the 6\textsuperscript{th} edition.\\

\noindent \textbf{Problem 5. Grouped Regression.}\\

\noindent Consider the linear model 
\begin{equation} 
y_{it} = \alpha_0 + \alpha_1 x_{it} + \varepsilon_{it}, 
\end{equation}
\noindent where the usual notation applies and $x_{it} \independent \varepsilon_{it}$. Note that $x_{it}$ is a scalar.  Suppose that $x_{it} = W_{kt}$, where $k$ indexes a ``group.'' That is,  $x_{it}$ does not vary at the individual $i$ level; instead, it varies at the group-$k$ level. Show that the OLS estimator of $\alpha_1$ is identical to the weighted least-squares estimator of the following ``grouped'' regression: 
\begin{equation} 
Y_{kt} = \alpha_0 + \alpha_1 W_{kt} + \varepsilon_{kt}, 
\end{equation}
\noindent where $Y_{kt}$ is the group-$k$ average of $y_{it}$ and the weight for group $k$ is the number of observations in the $k$ group, $n_{k}$.\\
 
\bibliographystyle{chicago}
\bibliography{econed}

\end{document}